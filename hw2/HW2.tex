\documentclass{amsart}

% PACKAGES

\usepackage{graphicx}
\usepackage{amsmath}
\usepackage{amsfonts}
\usepackage{amssymb,enumerate}
\usepackage{amsthm,stmaryrd}
\usepackage[all]{xy}
\usepackage{hyperref}

%\theoremstyle{definition}
%\newtheorem{exer}{Exercise}
\newcommand{\llll}{\mathcal{L}}
\newcommand{\bbr}{\mathbb{R}}
\newcommand{\bbc}{\mathbb{C}}
\newcommand{\bbz}{\mathbb{Z}}
\newcommand{\bbq}{\mathbb{Q}}
\newcommand{\bbn}{\mathbb{N}}
\newcommand{\be}{\mathbf{e}}
\newcommand{\ba}{\mathbf{a}}
\newcommand{\fm}{\mathfrak{m}}
\newcommand{\Hom}{\operatorname{Hom}}
\renewcommand{\ker}{\operatorname{Ker}}
\newcommand{\im}{\operatorname{Im}}
\newcommand{\xra}{\xrightarrow}
\newcommand{\wti}{\widetilde}

\theoremstyle{plain}
\newtheorem{lem}{Lemma}
\newtheorem{cor}[lem]{Corollary}
\newtheorem{prop}[lem]{Proposition}
\newtheorem{thm}[lem]{Theorem}
\newtheorem{conj}[lem]{Conjecture}
\newtheorem{intthm}{Theorem}
\renewcommand{\theintthm}{\Alph{intthm}}

\theoremstyle{definition}
\newtheorem{defn}[lem]{Definition}
\newtheorem{ex}[lem]{Example}
\newtheorem{question}[lem]{Question}
\newtheorem{questions}[lem]{Questions}
\newtheorem{problem}[lem]{Problem}
\newtheorem{disc}[lem]{Remark}
\newtheorem{rmk}[lem]{Remark}
\newtheorem{construction}[lem]{Construction}
\newtheorem{notn}[lem]{Notation}
\newtheorem{fact}[lem]{Fact}
\newtheorem{para}[lem]{}
\newtheorem{exer}[lem]{Exercise}
\newtheorem{remarkdefinition}[lem]{Remark/Definition}
\newtheorem{notation}[lem]{Notation}
\newtheorem{step}{Step}
\newtheorem{convention}[lem]{Convention}
\newtheorem*{Convention}{Convention}
\newtheorem{assumption}[lem]{Assumption}

\newcommand{\fmn}{F^{m\times n}}
\newcommand{\fnn}{F^{n\times n}}
\newcommand{\col}{\operatorname{Col}}
\newcommand{\row}{\operatorname{Row}}
\newcommand{\Span}{\operatorname{Span}}	
\newcommand{\rank}{\operatorname{rank}}	

\usepackage{setspace}
\doublespacing

\usepackage{kbordermatrix}% http://www.hss.caltech.edu/~kcb/TeX/kbordermatrix.sty

\begin{document}


\noindent MATH 8170,  \\
Fall 2016\\
HW 2\\
Shuai Wei \\
\
\textwidth 6.0in \oddsidemargin 0.0in

\noindent \textbf{1.} \\
Let $X_A(t)$ denote the number of organisms in state A at time $t$.\\
Let $X_B(t)$ denote the number of organisms in state B.\\
Then $X = \{X_A, X_B\}$ is a CTMC with sojourn time rate $\upsilon_{ij} = \alpha i + \beta j$ for each state $(i,j) \in E$, where the state space is 
\[E = \{(i,j)\}_{i,j\geq 0,i +j \geq 1} ,\]
where $i,j \geq 0$ and $i+j \geq 1 $.\\
The sojourn time $T_{ij}$ for the state $(i,j)$ is exponentially distributed with rate $\alpha i + \beta j$.\\
The transition probability is :
\[ P_{(i,j)(i+2,j-1)} = \frac{\beta j}{\alpha i + \beta j}; \]
\[ P_{(i,j)(i-1,j+1)} = \frac{\alpha i}{\alpha i + \beta j}; \]

\newpage

\noindent \textbf{2.}\\
Let $X(t)$ denote the number of customers in the system at time $t$. \\
The time until the next entering/birth $X \sim \text{exp}(\lambda \alpha_i)$, where $i$ is the number of customers already in the system.\\
The time until the next service ending/death $Y \sim \text{exp}(\mu)$.\\
The birth rate $\lambda \alpha_i $, the death rate is $\mu$.\\
Then $X(t)$ is a CTMC with sojourn time rate $\upsilon_i = \mu+ \lambda \alpha_i$ for state $i$.\\
Since it can only change its state by increasing by one or decreasing by one, it is a birth and death procee.\\
The state space is 
\[E= \{0,1,2,3,...\}.\]
The transition probability is :
\[ 
  	P_{ij}  =  \left\{ 
					\begin{array}{ll}
  					  P(X<Y), & j = i+1, i\geq 1 \\ 
					  P(Y<X),& j = i-1, i\geq 1  
					\end{array}
			 	  \right. 
  	           =  \left\{ 
					\begin{array}{ll}
  					  \frac{\lambda \alpha_i}{\mu + \lambda \alpha_i}, & j = i+1, i\geq 1 \\ 
					  \frac{\mu}{\mu+\lambda\alpha_i},& j = i-1, i\geq 1  
					\end{array}
			 	  \right. 
\]
When $i = 0$, $P_{01} = 1$.

\newpage

\noindent \textbf{3.}\\
Yes, it is a CTMC.\\
Assume the number of people who have a infection is $k$ given $1 \leq k \leq N$, then the state space is  
\[E = \{k, n+1,...,N\}.\]
For $1 \leq n \leq N$, let $\tau_n$ be the time when a new infection occurs.\\ 
When a contact occurs at time $t$ with $X(t) = i$ , the probability of one being infected is 
\[\frac{N(N-i)}{{N \choose 2}} = \frac{2i(N-i)}{N(N-1)}.\]  
Since contacts between two members of this population occur in accordance
with a Poisson process having rate $\lambda$,\\
the sojourn time $T_{n} = \tau_{n+1} - \tau_n$ for $X(\tau_n) = i$ with $k \leq i < N $and $1\leq n \leq N$ satisfies 
\[T_n \sim \text{exp}\left(\frac{2i(N-i)}{N(N-1)} \lambda \right).\]
For $k \leq i < N$, 
\begin{align*}
  P(X_{n+1} = i + 1, T_{n+1} > t \mid X_n = i, (X_j, T_j), j <n) &= P(X_{n+1} = i + 1, T_{n+1} > t \mid X_n = i) \\
  &= P(X_{1} = i + 1, T_{1} > t \mid X_0 = i) \\
  &= \text{exp}\left(-\frac{2i(N-i)}{N(N-1)} \lambda t\right).
\end{align*}
If the system enters the state $N$, it will stay there forever.

\newpage

\noindent \textbf{4.}\\
Define a CTMC with $E = \{0,1,2,3\}$.\\
The state $0$ means both machine 1 and 2 operate.\\
The state $1$ means machine 2 opertes but machine 1 is being repaired.\\
The state $2$ means machine 1 opertes but machine 2 is being repaired.\\
The state $3$ means both machine 1 and 2 are being repaired.\\
The operate time 
\[ O_i \sim \text{exp}(\lambda_i)\]  
for $i = 1,2$. \\
The repair time 
\[R_i \sim \text{exp}(\mu i)\]
for $i= 1,2$.\\
Then by the memoryless property of exponential distribution,
\[P_{01} = P(O_1 < O_2) = \frac{\lambda_1}{\lambda_1 + \lambda_2},\ \ 
P_{02} = P(O_2 < O_1) = \frac{\lambda_2}{\lambda_1 + \lambda_2},\]
\[P_{10} = P(R_1 < O_2) = \frac{\mu_1}{\mu_1 + \lambda_2},\ \ 
P_{13} = P(O_2 < R_1) = \frac{\lambda_2}{\mu_1 + \lambda_2},\]
\[P_{20} = P(R_2 < O_1) = \frac{\mu_2}{\lambda_1 + \mu_2}, \ \
P_{23} = P(O_1 < R_2) = \frac{\lambda_1}{\lambda_1 + \mu_2},\] 
\[P_{31} = P(R_2 < R_1) = \frac{\mu_2}{\mu_1+ \mu_2},\ \
P_{32} = P(R_1 < R_2) = \frac{\mu_1}{\mu_1+ \mu_2}.\]
Besides,
\[P_{03} = P_{30} = P_{12} = P_{21} = 0.\]
\renewcommand{\kbldelim}{(}% Left delimiter
\renewcommand{\kbrdelim}{)}% Right delimiter


We know
\[\upsilon_0 = \lambda_1 + \lambda_2,\  
\upsilon_1 = \mu_1 + \lambda_2,\
\upsilon_2 = \lambda_1 + \mu_2,\
\upsilon_3 = \mu_1 + \mu_2.\]
So
\[q_{01} = \upsilon_0P_{01} = \lambda_1, \
q_{02} = \upsilon_0P_{02} = \lambda_2,\]
\[q_{10} = \upsilon_1P_{10} = \mu_1, \
q_{13} = \upsilon_1P_{13} = \lambda_2, \]
\[q_{20} = \upsilon_2P_{20} = \mu_2, \
q_{23} = \upsilon_2P_{23} = \lambda_1, \]
\[q_{31} = \upsilon_3P_{31} = \mu_2, \
q_{32} = \upsilon_3P_{32} = \mu_1, \]
and
\[q_{03} = q_{30} = q_{12} = q_{21} = 0.\]
Moreover, $q_{ii} = \upsilon_i$ for $i = 0,1,2,3$.
Thus, the transition rate matrix is 
\[
    Q= \kbordermatrix{
    	& 0 & 1 & 2 & 3 \\
   	  0 & -(\lambda_1 + \lambda_2) & \lambda_1 & \lambda_2 & 0  \\
      1 &  \mu_1 & -(\mu_1 + \lambda_2) & 0 & \lambda_2  \\
      2 & \mu_2 & 0 & -(\lambda_1 + \mu_2) & \lambda_1  \\
      3 & 0 & \mu_2 & \mu_1 & -(\mu_1 + \mu_2)  \\
  }
\]
Next we compute the transition matrix. \\
Consider first the case that there is just the machine 1.\\
Definte a CTMC with state space $ E = \{o, r\}$.\\
State $o$ means it operates and $r$ means it is being repaired.\\
Then $\upsilon_{o} = \lambda_1$ and $\upsilon_{r} = \mu_1$.\\
So the transition rate matrix is 
\[
    Q_1= \kbordermatrix{
    	& o & r \\
      o & -\lambda_1 & \lambda_1\\
      r & \mu_1 & -\mu_1
  }
\]
According to the computational results from a similar example in class, we have \\
\[ P^1_{oo}(t) = \frac{\mu_1}{\lambda_1 + \mu_1} + \frac{\lambda_1}{\lambda_1 + \mu_1}e^{-(\lambda_1 + \mu_1)t},\
P^1_{or}(t) = \frac{\lambda_1}{\lambda_1 + \mu_1} - \frac{\lambda_1}{\lambda_1 + \mu_1}e^{-(\lambda_1 + \mu_1)t},\]
\[ P^1_{rr}(t) = \frac{\lambda_1}{\lambda_1 + \mu_1} - \frac{\mu_1}{\lambda_1 + \mu_1}e^{-(\lambda_1 + \mu_1)t},\
P^1_{ro}(t) = \frac{\mu_1}{\lambda_1 + \mu_1} - \frac{\mu_1}{\lambda_1 + \mu_1}e^{-(\lambda_1 + \mu_1)t}.\]
In addtion, we have the similar result for the machine $2$.\\
Since the machines act independently of each other,\\
we have
\[P_{00}(t) = P^1_{oo}(t)P^2_{oo}(t) = \left(\frac{\mu_1}{\lambda_1 + \mu_1} + \frac{\lambda_1}{\lambda_1 + \mu_1}e^{-(\lambda_1 + \mu_1)t}\right) 
  \left(\frac{\mu_2}{\lambda_2 + \mu_2} + \frac{\lambda_2}{\lambda_2+ \mu_2}e^{-(\lambda_2 + \mu_2)t}\right),
\]
\[P_{01}(t) = P^1_{or}(t)P^2_{oo}(t) = \left(\frac{\lambda_1}{\lambda_1 + \mu_1} - \frac{\lambda_1}{\lambda_1 + \mu_1}e^{-(\lambda_1 + \mu_1)t}\right) \left(\frac{\mu_2}{\lambda_2 + \mu_2} + \frac{\lambda_2}{\lambda_2 + \mu_2}e^{-(\lambda_2 + \mu_2)t}\right), \]
\[P_{02}(t) = P^1_{oo}(t)P^2_{or}(t) = \left(\frac{\mu_1}{\lambda_1 + \mu_1} + \frac{\lambda_1}{\lambda_1 + \mu_1}e^{-(\lambda_1 + \mu_1)t} \right)\left(\frac{\lambda_2}{\lambda_2 + \mu_2} - \frac{\lambda_2}{\lambda_2 + \mu_2}e^{-(\lambda_2 + \mu_2)t}\right),
\]
\[P_{03}(t) = P^1_{or}(t)P^2_{or}(t) =\left(\frac{\lambda_1}{\lambda_1 + \mu_1} - \frac{\lambda_1}{\lambda_1 + \mu_1}e^{-(\lambda_1 + \mu_1)t}\right)
\left(\frac{\lambda_2}{\lambda_2 + \mu_2} - \frac{\lambda_2}{\lambda_2 + \mu_2}e^{-(\lambda_2 + \mu_2)t}\right),\]
\[P_{10}(t) = P^1_{ro}(t)P^2_{oo}(t) = \left(\frac{\mu_1}{\lambda_1 + \mu_1} - \frac{\mu_1}{\lambda_1 + \mu_1}e^{-(\lambda_1 + \mu_1)t} \right)\left( \frac{\mu_2}{\lambda_2 + \mu_2} + \frac{\lambda_2}{\lambda_2 + \mu_2}e^{-(\lambda_2 + \mu_2)t}\right),
\]
\[P_{20}(t) = P^1_{oo}(t)P^2_{ro}(t) = \left(\frac{\mu_1}{\lambda_1 + \mu_1} + \frac{\lambda_1}{\lambda_1 + \mu_1}e^{-(\lambda_1 + \mu_1)t}\right)\left( \frac{\mu_2}{\lambda_2 + \mu_2} - \frac{\mu_2}{\lambda_2 + \mu_2}e^{-(\lambda_2 + \mu_2)t}\right),
\]
\[P_{30}(t) = P^1_{ro}(t)P^2_{ro}(t) =\left(\frac{\mu_1}{\lambda_1 + \mu_1} - \frac{\mu_1}{\lambda_1 + \mu_1}e^{-(\lambda_1 + \mu_1)t}\right)\left(\frac{\mu_2}{\lambda_2 + \mu_2} - \frac{\mu_2}{\lambda_2 + \mu_2}e^{-(\lambda_2 + \mu_2)t} \right). \]

Then the transition matrix 
\[
  P(t)= \kbordermatrix{
		& 0 & 1 & 2 & 3 \\
      0 &P^1_{or}P^2_{oo}& 
    	P^1_{or}P^2_{oo}&
    	P^1_{oo}P^2_{or}&
    	P^1_{or}P^2_{or}& \\
      1 &P^1_{ro}P^2_{oo}& 
    	P^1_{rr}P^2_{oo}&
    	P^1_{ro}P^2_{or}&
    	P^1_{rr}P^2_{or}& \\
      2 &P^1_{oo}P^2_{ro}& 
    	P^1_{or}P^2_{ro}&
    	P^1_{oo}P^2_{rr}&
    	P^1_{or}P^2_{rr}& \\
      3 &P^1_{ro}P^2_{ro}& 
    	P^1_{rr}P^2_{ro}&
    	P^1_{ro}P^2_{rr}&
    	P^1_{rr}P^2_{rr}& \\
    }
\]
\begin{align*}
  (QP)_{00} &=\sum_{k=0}^{3}q_{0k}P_{k0}(t) \\
  			&= q_{00}P_{00}(t)+q_{01}P_{10}(t)+q_{02}P_{20}(t)+q_{03}P_{30}(t) \\
  			&=-(\lambda_1 + \lambda_2) \left(\frac{\mu_1}{\lambda_1 + \mu_1} + \frac{\lambda_1}{\lambda_1 + \mu_1}e^{-(\lambda_1 + \mu_1)      t}\right)  \left(\frac{\mu_2}{\lambda_2 + \mu_2} + \frac{\lambda_2}{\lambda_2+ \mu_2}e^{-(\lambda_2 + \mu_2)t}\right) \\
  			&\ \ \ +\lambda_1 \left(\frac{\mu_1}{\lambda_1 + \mu_1} - \frac{\mu_1}{\lambda_1 + \mu_1}e^{-(\lambda_1 + \mu_1)t} \right)\left( \frac{\mu_2}{\lambda_2 + \mu_2} + \frac{\lambda_2}{\lambda_2 + \mu_2}e^{-(\lambda_2 + \mu_2)t}\right) \\
 		    &\ \ \ + \lambda_2\left(\frac{\mu_1}{\lambda_1 + \mu_1} + \frac{\lambda_1}{\lambda_1 + \mu_1}e^{-(\lambda_1 + \mu_1)      t}\right)\left( \frac{\mu_2}{\lambda_2 + \mu_2} - \frac{\mu_2}{\lambda_2 + \mu_2}e^{-(\lambda_2 + \mu_2)t}\right) \\
  			&=- \frac{\lambda_1\mu_2(\lambda_1+\mu_1)}{(\lambda_1+\mu_1)(\lambda_2+\mu_2)}e^{-(\lambda_1 + \mu_1)t} + \frac{\lambda_2\mu_1(\lambda_2+\mu_2)}{(\lambda_1+\mu_1)(\lambda_2+\mu_2)}e^{-(\lambda_2 + \mu_2)t}\\
 			&\ \ \ - \frac{\lambda_1\lambda_2(\lambda_1 +\lambda_2 +\mu_1+\mu_2)}{(\lambda_1+\mu_1)(\lambda_2+\mu_2)}e^{-(\lambda_1+\lambda_2 +\mu_1+ \mu_2)t} \\
  			&=-\frac{\lambda_1\mu_2}{(\lambda_2+\mu_2)}e^{-(\lambda_1 + \mu_1)t} - \frac{\lambda_2\mu_1}{\lambda_1+\mu_1}e^{-(\lambda_2 + \mu_2)t}\\
			& \ \ \ - \frac{\lambda_1\lambda_2(\lambda_1+\lambda_2 +\mu_1+ \mu_2)}{(\lambda_1+\mu_1)(\lambda_2+\mu_2)}e^{-(\lambda_1+\lambda_2+\mu_1+ \mu_2)t}.
\end{align*}
Besides,
\begin{align*}
  P_{00}\textprime (t) &= d\frac{P_{00}(t)}{dt} \\
   		&= d\frac{ \left(\frac{\mu_1}{\lambda_1 + \mu_1} + \frac{\lambda_1}{\lambda_1 + \mu_1}e^{-(\lambda_1 + \mu_1)      t}\right)
 	   \left(\frac{\mu_2}{\lambda_2 + \mu_2} + \frac{\lambda_2}{\lambda_2+ \mu_2}e^{-(\lambda_2 + \mu_2)t}\right)}{dt} \\
 	   &=-\lambda_1e^{-(\lambda_1 + \mu_1)t} \left(\frac{\mu_2}{\lambda_2 + \mu_2} + \frac{\lambda_2}{\lambda_2+ \mu_2}e^{-(\lambda_2 + \mu_2)t}\right) \\
 	   & \ \ \ - \lambda_2 e^{-(\lambda_2 + \mu_2)t} \left(\frac{\mu_1}{\lambda_1 + \mu_1} + \frac{\lambda_1}{\lambda_1 + \mu_1}e^{-(\lambda_1 + \mu_1)t}\right) \\
 	   &=-\frac{\lambda_1\mu_2}{(\lambda_2+\mu_2)}e^{-(\lambda_1 + \mu_1)t} - \frac{\lambda_2\mu_1}{\lambda_1+\mu_1}e^{-(\lambda_2 + \mu_2)t} \\
 	   & \ \ \ - \frac{\lambda_1\lambda_2(\lambda_1+\lambda_2 +\mu_1+ \mu_2)}{(\lambda_1+\mu_1)(\lambda_2+\mu_2)}e^{-(\lambda_1+\lambda_2 +\mu_1+ \mu_2)t}.
\end{align*}
Thus,
\[\sum_{k=0}^{3}q_{0k}P_{k0}(t) = P_{00}\textprime (t).\]
Similarly, for other $i,j \in E$, we have 
\[\sum_{k=0}^{3}q_{ik}P_{kj}(t) = P_{ij}\textprime (t).\]
As a result, the transition probability satisfies the forward equations.\\
\begin{align*}
  (PQ)_{00} &=\sum_{k=0}^{3}P_{0k}(t)q_{k0} \\
  			&= P_{00}(t)q_{00}+P_{01}(t)q_{10}+P_{02}(t)q_{20}+P_{03}(t)q_{30} \\
  			&=\left(\frac{\mu_1}{\lambda_1 + \mu_1} + \frac{\lambda_1}{\lambda_1 + \mu_1}e^{-(\lambda_1 + \mu_1)      t}\right)
  \left(\frac{\mu_2}{\lambda_2 + \mu_2} + \frac{\lambda_2}{\lambda_2+ \mu_2}e^{-(\lambda_2 + \mu_2)t}\right) \left(-(\lambda_1+\lambda_2)\right) \\
  			&\ \ \ +\left(\frac{\lambda_1}{\lambda_1 + \mu_1} - \frac{\lambda_1}{\lambda_1 + \mu_1}e^{-(\lambda_1 + \mu_1)t}\right) \left(\frac{\mu_2}{\lambda_2 + \mu_2} + \frac{\lambda_2}{\lambda_2 + \mu_2}e^{-(\lambda_2 + \mu_2)t}\right)\mu_1\\
  			&\ \ \ +\left(\frac{\mu_1}{\lambda_1 + \mu_1} + \frac{\lambda_1}{\lambda_1 + \mu_1}e^{-(\lambda_1 + \mu_1)      t} \right)\left(\frac{\lambda_2}{\lambda_2 + \mu_2} - \frac{\lambda_2}{\lambda_2 + \mu_2}e^{-(\lambda_2 + \mu_2)t}\right)\mu_2 + 0\\
			&=-\frac{\lambda_1\mu_2(\lambda_1+\mu_1)}{(\lambda_1+\mu_1)(\lambda_2+\mu_2)}e^{-(\lambda_1 + \mu_1)t} - \frac{\lambda_2\mu_1(\lambda_2+\mu_2)}{(\lambda_1+\mu_1)(\lambda_2+\mu_2)}e^{-(\lambda_2 + \mu_2)t} \\
  			&\ \ \ + \frac{\lambda_1\lambda_2(\lambda_1+\lambda_2 +\mu_1+ \mu_2)}{(\lambda_1+\mu_1)(\lambda_2+\mu_2)}e^{-(\lambda_1+\lambda_2+\mu_1+ \mu_2)t}\\
			&=-\frac{\lambda_1\mu_2}{(\lambda_2+\mu_2)}e^{-(\lambda_1 + \mu_1)t} - \frac{\lambda_2\mu_1}{\lambda_1+\mu_1}e^{-(\lambda_2 + \mu_      2)t} \\
            & \ \ \ - \frac{\lambda_1\lambda_2(\lambda_1+\lambda_2 +\mu_1+ \mu_2)}{(\lambda_1+\mu_1)(\lambda_2+\mu_2)}e^{-(\lambda_1+\lambda_2+\mu_1+ \mu_2)t}\\
  			&=P_{00}\textprime (t). 
\end{align*}
by previous steps.\\
Namely,
\[\sum_{k=0}^{3}P_{0k}(t)q_{k0} = P_{00}\textprime (t).\]

Similarly, for other $i,j \in E$, we have 
\[\sum_{k=0}^{3}P_{ik}(t)q_{kj} = P_{ij}\textprime (t).\]
As a result, the transition probability satisfies the backward equations.\\

\newpage
Let $X_{t}$ be the number of individuals at time $t$.\\
Assmue $X(0) = N \in bbn$.\\
Then the state space is $E=\{0,1,2,...,N\}$.\\
Then when $1 \leq i \leq $ and $i \in N$,
\[P_{ii-1} = 1.\]
When the system enters the state $0$, will stay there forever.\\
Then the transition rate
\[q_{ii} = -\mu\] 
for $1 \leq i \leq $ and $i \in N$.\\
Besides,
\[q_{ii-1} = \mu P_{ii-1} = mu\]
for $1 \leq i \leq $ and $i \in N$.\\





\end{document}
