\documentclass{amsart}

% PACKAGES

\usepackage{amsmath}
\usepackage{amsfonts}
\usepackage{amssymb,enumerate}
\usepackage{amsthm,stmaryrd}
\usepackage[all]{xy}
\usepackage{hyperref}

%\theoremstyle{definition}
%\newtheorem{exer}{Exercise}
\newcommand{\llll}{\mathcal{L}}
\newcommand{\bbr}{\mathbb{R}}
\newcommand{\bbc}{\mathbb{C}}
\newcommand{\bbz}{\mathbb{Z}}
\newcommand{\bbq}{\mathbb{Q}}
\newcommand{\bbn}{\mathbb{N}}
\newcommand{\be}{\mathbf{e}}
\newcommand{\ba}{\mathbf{a}}
\newcommand{\fm}{\mathfrak{m}}
\newcommand{\Hom}{\operatorname{Hom}}
\renewcommand{\ker}{\operatorname{Ker}}
\newcommand{\im}{\operatorname{Im}}
\newcommand{\xra}{\xrightarrow}
\newcommand{\wti}{\widetilde}

\theoremstyle{plain}
\newtheorem{lem}{Lemma}
\newtheorem{cor}[lem]{Corollary}
\newtheorem{prop}[lem]{Proposition}
\newtheorem{thm}[lem]{Theorem}
\newtheorem{conj}[lem]{Conjecture}
\newtheorem{intthm}{Theorem}
\renewcommand{\theintthm}{\Alph{intthm}}

\theoremstyle{definition}
\newtheorem{defn}[lem]{Definition}
\newtheorem{ex}[lem]{Example}
\newtheorem{question}[lem]{Question}
\newtheorem{questions}[lem]{Questions}
\newtheorem{problem}[lem]{Problem}
\newtheorem{disc}[lem]{Remark}
\newtheorem{rmk}[lem]{Remark}
\newtheorem{construction}[lem]{Construction}
\newtheorem{notn}[lem]{Notation}
\newtheorem{fact}[lem]{Fact}
\newtheorem{para}[lem]{}
\newtheorem{exer}[lem]{Exercise}
\newtheorem{remarkdefinition}[lem]{Remark/Definition}
\newtheorem{notation}[lem]{Notation}
\newtheorem{step}{Step}
\newtheorem{convention}[lem]{Convention}
\newtheorem*{Convention}{Convention}
\newtheorem{assumption}[lem]{Assumption}

\newcommand{\fmn}{F^{m\times n}}
\newcommand{\fnn}{F^{n\times n}}
\newcommand{\col}{\operatorname{Col}}
\newcommand{\row}{\operatorname{Row}}
\newcommand{\Span}{\operatorname{Span}}	
\newcommand{\rank}{\operatorname{rank}}	

\begin{document}

\noindent MATH 8170,  \\
Fall 2016\\
HW 2\\
Shuai Wei \\
\
\textwidth 6.0in \oddsidemargin 0.0in

\noindent \textbf{1.} 
Let $X_A(t)$ denote the number of organisms in state A at time $t$.\\
Let $X_B(t)$ denote the number of organisms in state B.\\
Then $X = \{X_A, X_B\}$ is a CTMC with sojourn time rate $\upsilon_{ij} = \alpha i + \beta j$ for state $(i,j)$. \\
The state space is 
\[E = \{(i,j)\}_{i,j\geq 0,i +j \geq 1} ,\]
where $i,j \geq 0$ and $i+j \geq 1 $.\\
The sojourn time $T_{ij}$ for the state $(i,j)$ is exponentially distributed with rate $\alpha i + \beta j$.\\
The transition probability is :
\[ P_{(i,j)(i+2,j-1)} = \frac{\beta j}{\alpha i + \beta j}; \]
\[ P_{(i,j)(i-1,j+1)} = \frac{\alpha i}{\alpha i + \beta j}; \]

\vspace{3mm}

\noindent \textbf{2.}
Let $X(t)$ denote the number of customers in the system at time $t$. \\
The time until the next entering/birth $\sim exp(\lambda \alpha_i)$.\\
The time until the next service ending/death $Y \sim exp(\mu)$.\\
The birth rate $\lambda \alpha_i $, the death rate is $\mu$.\\
Then $X(t)$ is a CTMC with sojourn time rate $\upsilon_i = \mu+ \lambda \alpha_i$ for state $i$.\\
Since it can only change its state by increasing by one or decreasing by one, it is a birth and death procee.\\
The state space is
\[ E= \{0,1,2,3,...\}\]
The transition probability is :
\[ 
  	P_{ij}  =  \left\{ 
					\begin{array}{ll}
  					  P(X<Y), & j = i+1, i\geq 1 \\ 
					  P(Y<X),& j = i-1, i\geq 1  
					\end{array}
			 	  \right. 
  	           =  \left\{ 
					\begin{array}{ll}
  					  \frac{\lambda \alpha_i}{\mu + \lambda \alpha_i}, & j = i+1, i\geq 1 \\ 
					  \frac{\mu}{\mu+\lambda\alpha_i},& j = i-1, i\geq 1  
					\end{array}
			 	  \right. 
\]
When $i = 0$, $P_{01} = 1$.
\end{document}
