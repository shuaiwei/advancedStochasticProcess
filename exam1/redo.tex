\documentclass{extarticle}
% PACKAGES

\usepackage{amsmath}
\usepackage{amsfonts}
\usepackage{amssymb,enumerate}
\usepackage{amsthm,stmaryrd}
\usepackage[all]{xy}
\usepackage{hyperref}

%\theoremstyle{definition}
%\newtheorem{exer}{Exercise}
\usepackage{bbold}
\newcommand{\idca}{\mathbb{1}}
\newcommand{\llll}{\mathcal{L}}
\newcommand{\bbr}{\mathbb{R}}
\newcommand{\bbc}{\mathbb{C}}
\newcommand{\bbz}{\mathbb{Z}}
\newcommand{\bbq}{\mathbb{Q}}
\newcommand{\bbn}{\mathbb{N}}
\newcommand{\be}{\mathbf{e}}
\newcommand{\ba}{\mathbf{a}}
\newcommand{\fm}{\mathfrak{m}}
\newcommand{\Hom}{\operatorname{Hom}}
\renewcommand{\ker}{\operatorname{Ker}}
\newcommand{\im}{\operatorname{Im}}
\newcommand{\xra}{\xrightarrow}
\newcommand{\wti}{\widetilde}

\theoremstyle{plain}
\newtheorem{lem}{Lemma}
\newtheorem{cor}[lem]{Corollary}
\newtheorem{prop}[lem]{Proposition}
\newtheorem{thm}[lem]{Theorem}
\newtheorem{conj}[lem]{Conjecture}
\newtheorem{intthm}{Theorem}
\renewcommand{\theintthm}{\Alph{intthm}}

\theoremstyle{definition}
\newtheorem{defn}[lem]{Definition}
\newtheorem{ex}[lem]{Example}
\newtheorem{question}[lem]{Question}
\newtheorem{questions}[lem]{Questions}
\newtheorem{problem}[lem]{Problem}
\newtheorem{disc}[lem]{Remark}
\newtheorem{rmk}[lem]{Remark}
\newtheorem{construction}[lem]{Construction}
\newtheorem{notn}[lem]{Notation}
\newtheorem{fact}[lem]{Fact}
\newtheorem{para}[lem]{}
\newtheorem{exer}[lem]{Exercise}
\newtheorem{remarkdefinition}[lem]{Remark/Definition}
\newtheorem{notation}[lem]{Notation}
\newtheorem{step}{Step}
\newtheorem{convention}[lem]{Convention}
\newtheorem*{Convention}{Convention}
\newtheorem{assumption}[lem]{Assumption}

\newcommand{\fmn}{F^{m\times n}}
\newcommand{\fnn}{F^{n\times n}}
\newcommand{\col}{\operatorname{Col}}
\newcommand{\row}{\operatorname{Row}}
\newcommand{\Span}{\operatorname{Span}}	
\newcommand{\rank}{\operatorname{rank}}	

\usepackage{setspace}

\doublespacing

\begin{document}

\begin{center}
    \textbf{Homework 4, MATH 8010}

	Shuai Wei 
\end{center}

\noindent \textbf{2.} \\
\begin{enumerate}[(b)]
	\item
	\begin{proof}
	  Let $T_n$ be the $n^{th}$ replacement for $n \geq 1$ and set $T_{0} = 0$.\\
	  Then $\{T_n-T_{n-1}\}_{n=1}^{\infty}$ is an $iid$ sequence of nonnegative random variables.
		Let $S_1$ be the first in-use failure.\\
		Let $N = \text{inf}\{n \geq 1: T_{n}-T_{n-1} < T\}$.\\
	  	Since $\idca(N \leq n) = 1-\idca(N > n) = 1-\idca(T_{1}-T_{0} \geq T, T_{2}-T_{1}> T,...,T_{n}-T_{n-1} \geq T)$, $N$ is a stopping time w.r.t. $\{T_n-T_{n-1}\}_{n=0}^{\infty}$.\\
	  	Then 
	  	\[S_1 = \sum_{n=1}^{N}(T_n-T_{n-1}).\]
	  	By Wald's identity, we have 
	  	\[E(S_1) = E(N)E(T_1).\]
		Since for $n \in \bbn$, 
		\begin{align*}
		  P(N=n) &=P(T_n-T_{n-1} < T)\prod_{k=1}^{n-1}P(T_k-T_{k-1} \geq T) \\
				 	&=P(T_1 < T)\prod_{k=1}^{n-1}P(T_1 \geq T) \\
		  			&=F(T)\left(1-F(T)\right)^{n-1}, 
	  	\end{align*}
		\[E(N) = \sum_{n=1}^{\infty}nP(N=n) = F(T)\sum_{n=1}^{\infty}n\left(1-F(T)\right)^{n-1}. \]
		We know the Maclaurin series of $\frac{1}{(1-x)^2}$ is 
		\[ \sum_{n=1}^{\infty}nx^{n-1}\] 
		for $-1 < x < 1$.\\
	    The density $f$ is continuous on $(0,\infty)$, so $0 < F(T) < 1$ for $T \in \bbr^+$.\\
	    Then we have $0 < 1-F(T) < 1$.\\
	    So, replace $x$ with $1-F(T)$, we have 
	    \[ \sum_{n=1}^{\infty}n\left(1-F(T)\right)^{n-1} = \frac{1}{1-\left(1-F(T)\right)^2} = \frac{1}{F^2(T)}. \] 
		Then 
	    \[E(N) = F(T)\frac{1}{F^2(T)} = \frac{1}{F(T)}.\]
	    Thus,the averge time between two successive in-use failures is
	    \[E(S_1) = E(N)E(T_1) = \frac{\int_0^Txf(x)dx+T\left(1-F(T)\right)}{F(T)}.\]
	\end{proof}
 \end{enumerate}
\newpage


\noindent \textbf{3.}\\
\begin{enumerate}[(a)]
	\item
	  Assume $p \neq q$.\\
	  Let $\{X_n\}$ be the winning of the $n^{th}$ bet for $n \geq 1$.\\
	  Then for $n \geq 1$, 
	  \[ X_n = \left\{ 
	  	  \begin{array}{cc}
	  	  	1,& \text{ with probability }p;\\
	  	  	-1,& \text{ with probability }1-p. 
	  	  \end{array}
	  	  	\right.
	  \]
	  So $\{X_n\}_{n=1}^{\infty}$ is an $iid$ sequence of nonnegative random variables.\\
	  Then $E(X_n) = p-q$.\\
	  Let $T=\text{inf}\{n\geq 1: \sum_{k=1}^{n}X_k = N \text{ or } \sum_{k=1}^{n}X_k = 0\}$.\\
	  Since 
	  \begin{align*}
	  	\idca(T \leq n) &= 1- \idca(T > n) \\
	  					&= 1-\idca(0< \sum_{k=1}^{1}X_k <N, 0< \sum_{k=1}^{2}X_k <N, ..., 0< \sum_{k=1}^{n}X_k <N),
	  \end{align*}
	  $T$ is a stopping time w.r.t $\{\sum_{k=1}^{n}X_k\}_{n=1}^{\infty}$.\\
	  Let $Y$ be the final winning of the gambler when he stops.\\
	  So given the initial fortune $i$ \$, 
	  \[ Y = \left\{ 
	  	  \begin{array}{cc}
	  	  	N-i,& \text{ with probability } P_i = \frac{1-(q/p)^i}{1-(q/p)^N};\\
	  	  	-i,& \text{ with probability } 1-P_i = \frac{(q/p)^i-(q/p)^N}{1-(q/p)^N}. 
	  	  \end{array}
	  	  	\right.
	  \]
	  So 
	  \begin{align*}
  			E\left( \sum_{n=1}^{T}X_n\right) = E(Y) = (N-i)P_i -i(1-P_i).
	  \end{align*}
	  By Wald's identity, we also have
	  \[E\left( \sum_{n=1}^{T}X_n\right) = E(T)E(X_1) = (p-q)E(T).\]
	 Thus,
	 \[E(T) = \frac{(N-i)P_i -i(1-P_i)}{p-q}.\]
	\item
		Assume $p=q$.\\
	  	Let $E_i(T)$ be $E(T)$ when given initial fortune is $i$ \$.\\
	  	Then 
	  	\begin{align*}
	  	  E_i(T) &= E\left(E_i(T|X_1)\right) \\
	  	  		 &= E(T|X_1 = 1) \frac{1}{2} + E(T|X_1 = -1)\frac{1}{2} \\
	  	  		 &=\frac{1}{2}\left(1+E_{i+1}(T)\right) + \frac{1}{2}\left(1+E_{i-1}(T)\right)\\
	  	  		 &=1+\frac{1}{2}E_{i+1}(T) + \frac{1}{2}E_{i-1}(T)		
	  	\end{align*} 
		Let $d_i = E_i(T) - E_{i-1}(T)$.\\
		Then $d_{i+1} = d_i - 2$.\\
		We can find that 
		\begin{align*}
		  \left( \begin{array}{c}
		  	d_{i+1} \\
		  	-2
		  \end{array}
		  \right)
		   &= 
		  \left( 
		  		\begin{array}{cc}
		  		  1& 1 \\
		  		  0& 1
		  		\end{array}
		  \right)
		  \left( \begin{array}{c}
		  	d_{i} \\
		  	-2
		  \end{array}
		  \right) \\
		  &= 
		  \left( 
		  		\begin{array}{cc}
		  		  1& 1 \\
		  		  0& 1
		  		\end{array}
		  \right)^2
		  \left( \begin{array}{c}
		  	d_{i-1} \\
		  	-2
		  \end{array}
			\right) \\
		&= \ldots \\
		&=
		  \left( 
		  		\begin{array}{cc}
		  		  1& 1 \\
		  		  0& 1
		  		\end{array}
		  	  \right)^{i}
		  \left( \begin{array}{c}
		  	d_{1} \\
			-2
		  \end{array}
		\right) 
		\end{align*}
	We claim 
	\begin{align*}
		\left( 
		  		\begin{array}{cc}
		  		  1& 1 \\
		  		  0& 1
		  		\end{array}
		  \right)^i
		  =
		 \left( 
		  		\begin{array}{cc}
		  		  1& i \\
		  		  0& 1
		  		\end{array}
		  \right).
	\end{align*}
	Basic case:
	\begin{align*}
		 \left( 
		  		\begin{array}{cc}
		  		  1& 1 \\
		  		  0& 1
		  		\end{array}
		  \right)^2 
		  =
		 \left( 
		  		\begin{array}{cc}
		  		  1& 2 \\
		  		  0& 1
		  		\end{array}
		  \right) 
	\end{align*}
	Assume
	\begin{align*}
		\left( 
		  		\begin{array}{cc}
		  		  1& 1 \\
		  		  0& 1
		  		\end{array}
		  \right)^i
		  =
		 \left( 
		  		\begin{array}{cc}
		  		  1& i \\
		  		  0& 1
		  		\end{array}
		  \right)
	\end{align*}
	Then 
	\begin{align*}
		\left( 
		  		\begin{array}{cc}
		  		  1& 1 \\
		  		  0& 1
		  		\end{array}
		  	  \right)^{i+1}
		 & =
		\left( 
		  		\begin{array}{cc}
		  		  1& 1 \\
		  		  0& 1
		  		\end{array}
		  	  \right)^i
		\left( 
		  		\begin{array}{cc}
		  		  1& 1 \\
		  		  0& 1
		  		\end{array}
		 \right) \\
		 &=
		 \left( 
		  		\begin{array}{cc}
		  		  1& i+1 \\
		  		  0& 1
		  		\end{array}
		  \right).
	\end{align*}
	So the claim holds.
	Then we get \\
		\begin{align*}
		  \left( \begin{array}{c}
		  	d_{i+1} \\
		  	-2
		  \end{array}
		  \right)
		   &= 
		  \left( 
		  		\begin{array}{cc}
		  		  1& i \\
		  		  0& 1
		  		\end{array}
		  \right)
		  \left( \begin{array}{c}
		  	d_{1} \\
		  	-2
		  \end{array}
		\right)
		\end{align*}
		So 
		\[d_{i+1} = d_1 -2 i.\]
		So
		\[ d_{N} = d_1-2(N-1).\]
		We know $d_1 = E_1(T)-E_0(T) = E_1(T)$, and \\
		$d_{N} = E_N(T)-E_{N-1}(T) = -E_{N-1}(T)$.\\
		Since $p=q$, by symmetry, we have 
		\[E_1(T) = E_{N-1}(T).\]
		So 
		\[d_N = -d_1.\]
		Combine it with $d_{N} = d_1-2(N-1)$, we have
		\[d_1 = N-1.\]
		Then 
		\[d_i = d_1 - 2(i-1) = N-2i+1.\]
		As a result, given intial fortune $i$ \$, we have 
		\begin{align*}
		  E(T)=E_i(T) &= d_i+E_{i-1}(T) \\
		  		&= d_i + d_{i-1} + E_{i-2}(T) \\
		  		 &=\ldots \\
				   &=d_i + d_{i-1} + \ldots + d_{1}+E_{0}(T)\\
		  		 &= (N-2i+1) + \left(N-2(i-1)+1\right) + \ldots + N-1 + 0\\
		  		 &= Ni-i^2.
		\end{align*}
\end{enumerate}

\newpage

\noindent \text{4.}
Consider the regenerating process w.r.t. the arrival times of the train.\\
For $n \geq 1$, let $T_n$ be the arrival time of the $n^{th}$ customer.\\
Then the interrenewals $\{T_{n}-T_{n-1}\}$ is a $iid$ sequence of nonnegative random variables.\\ 
Besides, we know $\left(T_{n}-T_{n-1}\right) \sim exp(\lambda)$ for $n \geq 1$, so $E(T_1) = \frac{1}{\lambda}$ given $T(0)= 0$.\\
Since 
\[T_N = \sum_{n=1}^N \left(T_{n}-T_{n-1}\right),\]
\[E(T_N) = \sum_{n=1}^NE\left(T_{n}-T_{n-1}\right) = NE(T_{1}) = \frac{N}{\lambda}.\] 
Let $\tau_1$ be the first arrival time of the train.\\
Then $\tau_1 = T_N + K$. \\
So \[ E(\tau_1) = E(T_N + K) = \frac{N}{\lambda}  + K. \]
Suppose each customer pays us money at a rate of $c$ per unit time.\\
So at $T_N$, the $n^{th}$ customer will pay us $c(T_N-T_n)$ for $1 \leq n \leq N$.\\
Let $R_1$ be the reward earned when between time 0 and $T_N$.\\
Then 
\[R_1 = \sum_{n=1}^{N}c(T_N-T_n).\]
Since for $1 \leq n \leq N$,
\[ E(T_N-T_n)  = E(T_{N-n}) = \frac{N-n}{\lambda},\]
\[E(R_1) = c\sum_{n=1}^{N}E(T_N-T_n) = c\sum_{n=1}^N \frac{N-n}{\lambda} = \frac{c}{2\lambda}N(N-1).\]
Let $R_2$ be the reward earned between $T_N$ and $\tau_1$.\\
Then 
\[R_2 = cNK + \sum_{l=1}^{N(K)}c(K-T_l).\]
Since given $N(K) = n$ for $n \in \bbn$, we have $T_{l} \sim $ Uniform$(0, K)$ for $1 \leq l \leq n$,\\
\[E\left(T_l\right) = \int_{0}^K x \frac{1}{K} dx = \frac{K}{2}.\]
So
\begin{align*}
	E\left(\sum_{l=1}^{N(K)}c(K-T_l) \mid N(K) =n \right) &= E\left(\sum_{l=1}^{n}c(K-T_l) \right) \\
  								&= c\sum_{l=1}^nE(K-T_l) \\
  								&= c\sum_{l=1}^n\left(K-E(T_l)\right)\\
  								&= cKn-c\sum_{l=1}^nE(T_l) \\
  								& = cKn-c\sum_{l=1}^n \frac{K}{2} \\
  								&= \frac{cKn}{2}.
\end{align*}
Thus,
\begin{align*}
  E\left(\sum_{l=1}^{N(K)}c(K-T_l)\right) &= E\left( E\left(\sum_{l=1}^{N(K)}c(K-T_l) \mid N(K) =n \right)\right)\\
  										&= E\left(\frac{cKN(K)}{2} \right)\\
 			 							&= \frac{cK}{2}E\left(N(K)\right) \\
  										&= \frac{cK}{2} \lambda K \\
  									    &= \frac{c\lambda K^2}{2}
\end{align*}
since given $K$, $N(K)\sim$Poisson$(\lambda K)$.\\
So
\[E(R_2) = cNK + E\left( \sum_{l=1}^{N(K)}c(K-T_l) \right) = cNK + \frac{c\lambda K^2}{2}.\]
Since the total cost during the first cycle $\tau_1$ is 
\[C_1 = R_1 + R_2,\]
\[E(C_1) = E(R_1) + E(R_2) = \frac{c}{2\lambda}N(N-1) + cNK + \frac{c\lambda K^2}{2}.\]
As a result, the long-run average cost per unit time is 
\begin{align*}
	\frac{E(C_1)}{E(\tau_1)} &= \frac{\frac{c}{2\lambda}N(N-1) + cNK + \frac{c\lambda K^2}{2}}{\frac{N}{\lambda}  + K}\\
  							 &= \frac{cN(N-1) + 2c\lambda NK + c\lambda^2 K^2}{2N + 2\lambda K}.
\end{align*}
\end{document}


