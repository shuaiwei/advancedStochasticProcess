\documentclass[12pt]{extarticle}
% PACKAGES

\usepackage{amsmath}
\usepackage{amsfonts}
\usepackage{amssymb,enumerate}
\usepackage{amsthm,stmaryrd}
\usepackage[all]{xy}
\usepackage{hyperref}

%\theoremstyle{definition}
%\newtheorem{exer}{Exercise}
\usepackage{bbold}
\newcommand{\idca}{\mathbb{1}}
\newcommand{\llll}{\mathcal{L}}
\newcommand{\bbr}{\mathbb{R}}
\newcommand{\bbc}{\mathbb{C}}
\newcommand{\bbz}{\mathbb{Z}}
\newcommand{\bbq}{\mathbb{Q}}
\newcommand{\bbn}{\mathbb{N}}
\newcommand{\be}{\mathbf{e}}
\newcommand{\ba}{\mathbf{a}}
\newcommand{\fm}{\mathfrak{m}}
\newcommand{\Hom}{\operatorname{Hom}}
\renewcommand{\ker}{\operatorname{Ker}}
\newcommand{\im}{\operatorname{Im}}
\newcommand{\xra}{\xrightarrow}
\newcommand{\wti}{\widetilde}

\theoremstyle{plain}
\newtheorem{lem}{Lemma}
\newtheorem{cor}[lem]{Corollary}
\newtheorem{prop}[lem]{Proposition}
\newtheorem{thm}[lem]{Theorem}
\newtheorem{conj}[lem]{Conjecture}
\newtheorem{intthm}{Theorem}
\renewcommand{\theintthm}{\Alph{intthm}}

\theoremstyle{definition}
\newtheorem{defn}[lem]{Definition}
\newtheorem{ex}[lem]{Example}
\newtheorem{question}[lem]{Question}
\newtheorem{questions}[lem]{Questions}
\newtheorem{problem}[lem]{Problem}
\newtheorem{disc}[lem]{Remark}
\newtheorem{rmk}[lem]{Remark}
\newtheorem{construction}[lem]{Construction}
\newtheorem{notn}[lem]{Notation}
\newtheorem{fact}[lem]{Fact}
\newtheorem{para}[lem]{}
\newtheorem{exer}[lem]{Exercise}
\newtheorem{remarkdefinition}[lem]{Remark/Definition}
\newtheorem{notation}[lem]{Notation}
\newtheorem{step}{Step}
\newtheorem{convention}[lem]{Convention}
\newtheorem*{Convention}{Convention}
\newtheorem{assumption}[lem]{Assumption}

\newcommand{\fmn}{F^{m\times n}}
\newcommand{\fnn}{F^{n\times n}}
\newcommand{\col}{\operatorname{Col}}
\newcommand{\row}{\operatorname{Row}}
\newcommand{\Span}{\operatorname{Span}}	
\newcommand{\rank}{\operatorname{rank}}	

\usepackage{setspace}

\doublespacing

\begin{document}

\begin{center}
    \textbf{Homework 4, MATH 8010}

	Shuai Wei 
\end{center}

\noindent \textbf{2.} \\
\begin{enumerate}[(b)]
	\item
	\begin{proof}
	  Let $T_n$ be the $k^{th}$ replacement for $k \geq 1$ and set $T_{0} = 0$.\\
	  Then $\{T_n-T_{n-1}\}_{n=1}^{\infty}$ is an $iid$ sequence of nonnegative random variables.
		Let $S_1$ be the first in use failure.\\
		Let $N = \text{inf}\{n \geq 1: T_{n}-T_{n-1} < T\}$.\\
	  	Since $\idca(N \leq n) = 1-\idca(N > n) = 1-\idca(T_{1}-T_{0} \geq T, T_{2}-T_{1}> T,...,T_{n}-T_{n-1} \geq T)$, $N$ is a stopping time w.r.t. $\{T_n-T_{n-1}\}_{n=0}^{\infty}$.\\
	  	Then 
	  	\[S_1 = \sum_{n=1}^{N}(T_n-T_{n-1}).\]
	  	By Wald's identity, we have 
	  	\[E(S_1) = E(N)E(T_1).\]
		Since for $n \in \bbn$, 
		\begin{align*}
		  P(N=n) &=P(T_n-T_{n-1} < T)\prod_{k=1}^{n-1}P(T_k-T_{k-1} \geq T) \\
				 	&=P(T_1 < T)\prod_{k=1}^{n-1}P(T_1 \geq T) \\
		  			&=F(T)\left(1-F(T)\right)^{n-1}, 
	  	\end{align*}
		\[E(N) = \sum_{n=1}^{\infty}nP(N=n) = F(T)\sum_{n=1}^{\infty}n\left(1-F(T)\right)^{n-1}. \]
		We know the Maclaurin series of $\frac{1}{(1-x)^2}$ is 
		\[ \sum_{n=1}^{\infty}nx^{n-1}\] 
		for $-1 < x < 1$.\\
	    The density $f$ is continuous on $(0,\infty)$, so $0 < F(T) < 1$ for $T \in \bbr^+$.\\
	    Then we have $0 < 1-F(T) < 1$.\\
	    So, replace $x$ with $1-F(T)$, we have 
	    \[ \sum_{n=1}^{\infty}n\left(1-F(T)\right)^{n-1} = \frac{1}{1-\left(1-F(T)\right)^2} = \frac{1}{F^2(T)}. \] 
		Then 
	    \[E(N) = F(T)\frac{1}{F^2(T)} = \frac{1}{F(T)}.\]
	    Thus,the averge time between two successive in-use failures is
	    \[E(S_1) = E(N)E(T_1) = \frac{\int_0^Txf(x)dx+T\left(1-F(T)\right)}{F(T)}.\]

	\end{proof}
\end{enumerate}

\end{document}


